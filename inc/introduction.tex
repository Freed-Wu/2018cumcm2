\ifnum\strcmp{\jobname}{introduction}=0
	\input{../set/preclass}
	\documentclass{article}
	\input{../set/default}
	\input{../set/user}
	\makeatletter
%-------------------------------------------------------------------------------------------------------
\usepackage{enumitem}
\setlist[enumerate, 1]
{	fullwidth,
	label = \arabic*., 
	font = \textup, 
	itemindent=2em
}
\setlist[enumerate, 2]
{	fullwidth,
	label = \arabic*., 
	font = \textup, 
	itemindent=2em
}
%-------------------------------------------------------------------------------------------------------
\pagestyle{fancy}
\renewcommand{\headrulewidth}{0pt}
\lhead{}
\chead{}
\rhead{}
\lfoot{\small{第\chinese{section}节}}
\cfoot{\small{第\thepage 页~共\pageref{LastPage}页}}
\rfoot{}
%-------------------------------------------------------------------------------------------------------
\title{\textbf{突发性灾难应急医疗救援策略}}
\author{}
\date{}
%-------------------------------------------------------------------------------------------------------
\makeatother

	\begin{document}
\fi
%-------------------------------------------------------------------------------------------------------
\clearpage
\setcounter{section}{-1}
\section{引言}
\subsection{问题背景}
突发性灾难的医疗救援都是广大医护人员所肩负的重要职责,受伤人员就是命令,时间就是生命。突发性灾难医疗救护的第一个环节是对受伤人员进行检伤和分类,负责分类的医护人员应能在1~10分钟内根据伤因、伤情以及受伤人员的生命体征准确地完成受伤人员的初步分类,同时,应根据分类结果将受伤人员护送至相应的急救区。通常可以将重受伤人员分为“紧急组”和“非紧急组”两类,分别记为A类和B类。对于“紧急组”的重受伤人员,应立即给予有预见性的各种护理,并尽快进行复苏或手术治疗,保证在“伤后黄金1小时”内获得救治的机会。
\\\indent 在实际中,根据每一个突发性灾难医疗救援事件,通常都会设置相应的医疗救援中心,一般救援中心包含若干个救治单元,每个救治单元通常由一名医生和若干名护士组成。救治单元是应急医疗救援中的最小行动单位,救治单元的职责就是对受伤人员实施紧急手术或其他必要的紧急治疗。
\subsection{问题信息}
\begin{enumerate}
	\item 对应于A,B两类受伤人员,将救治单元也分为A,B两类,同类的各个救治单元的功能相同。根据经验,A类救治单元治疗一个A类受伤人员平均需要\(35\pm7\)(分钟),而治疗一个B类受伤人员平均需要\(40\pm5\)(分钟);B类救治单元治疗一个B类受伤人员平均需要\(30\pm6\)(分钟),而治疗一个A类受伤人员平均需要\(45\pm5\)(分钟)。
\end{enumerate}
\subsection{问题重述}
\begin{enumerate}
	\item 附件工作表1中给出了某次医疗救援事件中两类受伤人员到达的时间,现由10个A类救治单元和6个B类救治单元组成医疗救援中心。请你们给出两类受伤人员最佳的救治策略,并对受伤人员的等待救治时间和各救治单元的工作强度等方面给出评价。
	\item 如果将2个救治单元合为一组协作工作,在同时救治两个受伤人员的情况下,能使救治效率各提高10\%。按此方法再请你们给出附件工作表1中两类受伤人员最佳的救治策略,并对受伤人员的等待救治时间和各救治单元的工作强度等方面给出评价。
	\item 附件中工作表2和3分别给出了某次重大救援行动中A类和B类受伤人员陆续到达某救援中心的时间。在保证每个受伤人员等待时间尽量不超过30分钟的条件下,给出你们认为最合理的A和B两类救治单元的设置数量与救治策略,并对救治结果给出评价分析。
	\item 根据未来可能发生突发性灾难医疗救援,假设A,B两类受伤人员的到达分别服从于参数为\(\lambda_1,\lambda_2(\lambda_2\geqslant\lambda_1>0)\)的Poisson流。请你们给出任意时刻的救治策略,通过仿真说明救治策略的可行性和有效性;并请你们为决策机构做出一个应急医疗救援的预案。
\end{enumerate}
\subsection{问题分析}
\begin{enumerate}
	\item 问题一可分为两个过程,即排队过程和评价过程。
	\\排队过程
	\\考虑到救治单元和受伤人员的搭配情况,共有四种可能,最优的即为同类搭配,为了使排队时间最短,即为治疗时间最短,遵循如下原则:
	\begin{enumerate}
		\item 若治疗单元有空余,则遵循同类救治单元治疗同类病人的原则,至到治疗单元满员。
		\item 若治疗单元有空余,且此时无同类病人,则检查是否有不同类病人在等待,如果有有,则治疗此病人。
		\item 若治疗单元空余且无人等待,则等待下一个人到来。
		\item 若治疗单元满员,则等待治疗单元空余再按上述原则入队。
	\end{enumerate}
	\par 评价过程 
	\\将受伤人员从到救治中心到进入治疗单元的时间认为是受伤人员的等待救治时间,我们取人均的等待救治时间作为指标,而将各治疗单元的治疗时间和作为各救治单元的工作强度指标。 
	\item 我们将此题分为三个阶段,一是对两个救治单元合并为一组的分类,二是对合并后的新的救治系统进行排队模拟。三是进行比较选出最好的系统。
	\\合并的情况 由于两两合并一定会提高效率,因此让所有救治单元进行两两合并,共有如下几种情况:(1)5个AA,3个BB(2)2个AB、4个AA,2个BB(3)4个AB、3个AA,1个BB(4)6个AB、2个AA,
	\\排队模拟 运用问题一排队方法。
	\\合并的选择。
	\\根据排队模拟的结果,选取等待救治时间最小的组别作为最终合并组合。
	\item 题三共有三个关键点,即为救治单元设置数量,救治策略对救治结果进行分析
	\item 由于A、B为Poisson流,符合经典的排队论规则,便应用排队论进行救治单元数量的选择和对救治策略的评价。
	\\排队论的基本问题是研究一些数量指标在瞬时或平稳状态下的概率分布及其数字特征,了解系统运行的基本特征;系统数量指标的统计推断和系统的优化问题等。
	\\救治单元的选择 从一开始进行试探,根据排队论对救治策略的评价指标,按照某一规则确定处救治单元的个数,救治单元既要对病人有利,也要对医院有利。
\end{enumerate}
%-------------------------------------------------------------------------------------------------------
\ifnum\strcmp{\jobname}{introduction}=0
	\end{document}
\fi
