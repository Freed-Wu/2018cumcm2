\ifnum\strcmp{\jobname}{evaluate}=0
	\input{../set/preclass}
	\documentclass{article}
	\input{../set/default}
	\input{../set/user}
	\makeatletter
%-------------------------------------------------------------------------------------------------------
\usepackage{enumitem}
\setlist[enumerate, 1]
{	fullwidth,
	label = \arabic*., 
	font = \textup, 
	itemindent=2em
}
\setlist[enumerate, 2]
{	fullwidth,
	label = \arabic*., 
	font = \textup, 
	itemindent=2em
}
%-------------------------------------------------------------------------------------------------------
\pagestyle{fancy}
\renewcommand{\headrulewidth}{0pt}
\lhead{}
\chead{}
\rhead{}
\lfoot{\small{第\chinese{section}节}}
\cfoot{\small{第\thepage 页~共\pageref{LastPage}页}}
\rfoot{}
%-------------------------------------------------------------------------------------------------------
\title{\textbf{突发性灾难应急医疗救援策略}}
\author{}
\date{}
%-------------------------------------------------------------------------------------------------------
\makeatother

	\begin{document}
\fi
%-------------------------------------------------------------------------------------------------------
\clearpage
\section{模型的评价与改进}
\subsection{优点}
\begin{enumerate}
	\item 将逼近全局最优的贪心算法获得的结果当作模拟退火算法的初值,使解的可靠性更高,迭代次数相对减少,节约大量的时间。
	\item 由于模拟退火算法可以以一定概率获得全局最优解,这样会避免因为按照一定规律单步排序陷入局部最优的情况。
	\item 模拟退火算法具有并行性,可在较短时间得出较为精准的解。	
\end{enumerate}
\subsection{缺点}
\begin{enumerate}
	\item 贪心算法只考虑当前过程时间最短而未考虑下个病人的到来是否会比当前过程时间还要短,具有片面性。
	\item 模拟退火算法得到全局最优需要一定的概率,且其过程是不定的,其结果并每次均不相同,存在一定的风险。
	\item 假设时治疗时间在范围内符合正态分布,但正态分布不封闭,总会有一定概率超出题目所给范围。
	\item 第二问两两合并采用穷举法,,势必造成大量时间的浪费。
\end{enumerate}
\subsection{改进}
\begin{enumerate}
	\item 在第一步贪心算法的计算时,除了考虑当前病人所用时间最短,还要考虑如果下一个到来人员先治疗会不会时间更短。
	\item 对模拟退火算法进行改进,使其迭代次数足够多并记录每次的值,最后在这些值中找到最优值。
	\item 假设病人的时间符合范围内的均匀分布。
	\item 对组合方式进行优化,减少最终迭代时间。
\end{enumerate}
\subsection{推广}
\begin{enumerate}
	\item 改善就医流程、合理利用资源、提高就诊效率已成为医改的核心任务和各大医院面临的重要课题。\cite{2} 因此可以建立医疗服务排队管理系统。
	\item 服务业中的排队现象是非常普遍的。银行中需要排队办理业务,医院内需要排队挂号、分诊、取药、化验、检查和等床,\cite{3} 超市中需排队结账,旅游景点需要排队买门票,因此在进行相应更改后,排队模型可以推广到生活中的各种领域和方面。
\end{enumerate}
%-------------------------------------------------------------------------------------------------------
\ifnum\strcmp{\jobname}{evaluate}=0
	\end{document}
\fi
