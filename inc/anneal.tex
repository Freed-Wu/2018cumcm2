\ifnum\strcmp{\jobname}{greedy}=0
	\input{../set/preclass}
	\documentclass{article}
	\input{../set/default}
	\input{../set/user}
	\makeatletter
%-------------------------------------------------------------------------------------------------------
\usepackage{enumitem}
\setlist[enumerate, 1]
{	fullwidth,
	label = \arabic*., 
	font = \textup, 
	itemindent=2em
}
\setlist[enumerate, 2]
{	fullwidth,
	label = \arabic*., 
	font = \textup, 
	itemindent=2em
}
%-------------------------------------------------------------------------------------------------------
\pagestyle{fancy}
\renewcommand{\headrulewidth}{0pt}
\lhead{}
\chead{}
\rhead{}
\lfoot{\small{第\chinese{section}节}}
\cfoot{\small{第\thepage 页~共\pageref{LastPage}页}}
\rfoot{}
%-------------------------------------------------------------------------------------------------------
\title{\textbf{突发性灾难应急医疗救援策略}}
\author{}
\date{}
%-------------------------------------------------------------------------------------------------------
\makeatother

	\begin{document}
\fi
%-------------------------------------------------------------------------------------------------------
\clearpage
\section{退火算法}
\subsection{模型的建立与求解}
\subsubsection{建立}
每种病人都有送入A和B2种病房,所以依靠穷举算法时间复杂度达到了指数级别。考虑启发式算法。
\\\indent 模拟退火算法来源于固体退火原理,将固体加温至充分高,再让其徐徐冷却,加温时,固体内部粒子随温升变为无序状,内能增大,而徐徐冷却时粒子渐趋有序,在每个温度都达到平衡态,最后在常温时达到基态,内能减为最小。其在降温时,由于分子的热运动影响,其内能不一定会减小。会进行波动,此波动可以跳出局部最优解,以一定概率获得全局最优解。本题可以用此算法跳出贪心算法的局部最优解。
\subsubsection{求解}
随机选取若干个初始点进行迭代,结果稳定后停止迭代。其中最优的结果如下。
\subsection{模型的验证与分析}
\subsubsection{分析}
该解仍不一定是最优解,但相比贪心算法,更可能得到更优的解。
%-------------------------------------------------------------------------------------------------------
\ifnum\strcmp{\jobname}{greedy}=0
	\end{document}
\fi
