\ifnum\strcmp{\jobname}{abstract}=0
	\input{../set/preclass}
	\documentclass{article}
	\input{../set/default}
	\input{../set/user}
	\makeatletter
%-------------------------------------------------------------------------------------------------------
\usepackage{enumitem}
\setlist[enumerate, 1]
{	fullwidth,
	label = \arabic*., 
	font = \textup, 
	itemindent=2em
}
\setlist[enumerate, 2]
{	fullwidth,
	label = \arabic*., 
	font = \textup, 
	itemindent=2em
}
%-------------------------------------------------------------------------------------------------------
\pagestyle{fancy}
\renewcommand{\headrulewidth}{0pt}
\lhead{}
\chead{}
\rhead{}
\lfoot{\small{第\chinese{section}节}}
\cfoot{\small{第\thepage 页~共\pageref{LastPage}页}}
\rfoot{}
%-------------------------------------------------------------------------------------------------------
\title{\textbf{突发性灾难应急医疗救援策略}}
\author{}
\date{}
%-------------------------------------------------------------------------------------------------------
\makeatother

	\begin{document}
\fi
%-------------------------------------------------------------------------------------------------------
\setcounter{page}{1}
\maketitle
\thispagestyle{empty}
\begin{abstract}
	本文利用排队论的数学方法对突发性灾难的医疗救援进行分析,寻求能使伤病员等待救治时间最短的最优解决方案。
	\\\indent 针对问题一,先考虑了使得每个伤病员当前的等待时间最短的贪心算法。得到的解为平均等待10分钟,最长等待46分钟。之后考虑模拟退火算法,通过尝试不同的初始值求得可能的最优解;
	\\\indent 问题二并未给出2个救治中心协作的具体信息;如果没有限制,则与第1问解法相同;
	\\\indent 问题三增加了限制条件,要求每个伤病员等待时间不能超过30分钟,寻找满足这一条件的解;
	\\\indent 问题四没有给出具体输入,即伤病员到达的时间和种类,但却给出了其分布规律。先通过泊松分布的随机变量生成几组输入,再仿照问题一进行求解。	
	\\\textbf{关键词:}排队论;贪心算法;医疗救援。
\end{abstract}
\clearpage
%\begin{center}
%	\normalsize\textbf{Summary}
%\end{center}
%\par\small
%\begin{spacing}{0.5}
%	\\\textbf{Keywords: }
%\end{spacing}
%\normalsize
%\clearpage
%-------------------------------------------------------------------------------------------------------
\ifnum\strcmp{\jobname}{abstract}=0
	\end{document}
\fi
